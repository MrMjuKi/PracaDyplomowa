\pdfbookmark[0]{Streszczenie}{streszczenie.1}
%\phantomsection
%\addcontentsline{toc}{chapter}{Streszczenie}
%%% Poni�sze zosta�o niewykorzystane (tj. zrezygnowano z utworzenia nienumerowanego rozdzia�u na abstrakt)
%%%\begingroup
%%%\setlength\beforechapskip{48pt} % z jakiego� powodu by�a male�ka r�nica w po�o�eniu nag��wka rozdzia�u numerowanego i nienumerowanego
%%%\chapter*{\centering Abstrakt}
%%%\endgroup
%%%\label{sec:abstrakt}
%%%Lorem ipsum dolor sit amet eleifend et, congue arcu. Morbi tellus sit amet, massa. Vivamus est id risus. Sed sit amet, libero. Aenean ac ipsum. Mauris vel lectus. 
%%%
%%%Nam id nulla a adipiscing tortor, dictum ut, lobortis urna. Donec non dui. Cras tempus orci ipsum, molestie quis, lacinia varius nunc, rhoncus purus, consectetuer congue risus. 
%\mbox{}\vspace{2cm} % mo�na przesun��, w zale�no�ci od d�ugo�ci streszczenia
\begin{abstract}
Streszczenie w j�zyku polskim powininno zmie�ci� si� na po�owie strony (drug� po�ow� powinien zaj�� abstract w j�zyku angielskim).

Lorem ipsum dolor sit amet eleifend et, congue arcu. Morbi tellus sit amet, massa. Vivamus est id risus. Sed sit amet, libero. Aenean ac ipsum. Mauris vel lectus. 

Nam id nulla a adipiscing tortor, dictum ut, lobortis urna. Donec non dui. Cras tempus orci ipsum, molestie quis, lacinia varius nunc, rhoncus purus, consectetuer congue risus. 


\end{abstract}
\mykeywords{raz, dwa, trzy, cztery}

% Dobrze by�oby skopiowa� s�owa kluczowe do metadanych dokumentu pdf (w pliku Dyplom.tex)
% Niestety, zaimplementowane makro nie robi tego z automatu, wi�c pozostaje kopiowanie r�czne.

{
\selectlanguage{english}
\begin{abstract}
Streszczenie in Polish should fit on the half of the page (the other half should be covered by the abstract in English). 

Lorem ipsum dolor sit amet eleifend et, congue arcu. Morbi tellus sit amet, massa. Vivamus est id risus. Sed sit amet, libero. Aenean ac ipsum. Mauris vel lectus. 

Nam id nulla a adipiscing tortor, dictum ut, lobortis urna. Donec non dui. Cras tempus orci ipsum, molestie quis, lacinia varius nunc, rhoncus purus, consectetuer congue risus. 
\end{abstract}
\mykeywords{one, two, three, four}
}
