\section{Kod aplikacji i baza danych}
Zar�wno kod aplikacji oraz baza danych s� dost�pne w publicznym repozytorium pod nast�puj�cym~\href{https://github.com/MrMjuKi/Kod-aplikacji}{linkiem}. 

W pierwszej kolejno�ci, po pobraniu repozytorium, wymagane jest zaimportowanie bazy damych za pomoc� narz�dzia MySQL Workbench. Klikamy w zak�adk� \texttt{Server}, a nast�pnie \texttt{Data Import}. Wybieramy folder \texttt{baza\_danych\_ofert}. Zaznaczamy wszystkie mo�liwe pliki do zaimportowania oraz wpisujemy nazw� bazy \texttt{baza\_danych\_ofert}. Naciskamy \texttt{Start Import} i pomy�lnie zako�czono proces importowania.

W drugiej kolejno�ci musimy uruchomi� kod aplikacji w narz�dziu kompiluj�cym j�zyk programowania Python 3.9. Poni�szy przypadek opisuje jak to zrobi� za pomoc� narz�dzia Visual Studio 2022. Po jego uruchomieniu z zak�adki wybieramy \texttt{Pliki}, nast�pnie \texttt{Otw�rz plik} i~wybieramy plik o~nazwie \texttt{WebProcet/app.py}. Jest to g��wny plik aplikacji, kompilujemy go. Pochwili aplikacja uruchamia si� w nowej zak�adce przegl�darki.